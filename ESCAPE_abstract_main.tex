%&pdflatex
\documentclass[3p,review,authoryear,10pt]{elsarticle}

\usepackage{graphicx}
\usepackage{natbib}
\usepackage{setspace}
\usepackage{multirow}
\usepackage{booktabs}
\usepackage[T1]{fontenc}
%\setlength{\parindent}{0in}
\usepackage{times}
\usepackage[british]{babel}
\usepackage{algorithm}
\usepackage{algorithmic} \renewcommand{\algorithmicrequire}{\textbf{Given:}} \renewcommand{\algorithmicensure}{\textbf{Returns:}}
\usepackage{amsfonts}
\usepackage{mathrsfs}
\usepackage{amssymb}
\usepackage{amsmath}
\usepackage{mathtools}
\renewcommand \thesection{\arabic{section}}
\usepackage[caption=false]{subfig}
\usepackage[compact]{titlesec}
\usepackage{soul}
\usepackage{textcomp}
\usepackage[section]{placeins}
\usepackage{url}
\usepackage{xspace}
%\usepackage{lineno}
\usepackage{hyperref}
    \hypersetup{colorlinks=true,linkcolor=green}


%Added by Diarmid
\usepackage{siunitx}
\usepackage{setspace}
\doublespacing
\usepackage{cleveref}


% % new commands
\newcommand{\I}{\mathcal{I}}
\DeclareMathOperator*{\any}{any}
\DeclareMathOperator*{\all}{all}

\newcommand{\ddt}[1]{\frac{\partial #1}{\partial t}}
\newcommand{\ddx}[1]{\frac{\partial #1}{\partial x}}

%% for annotating paper for TODO notes etc.  Comment this out and
%% uncomment alternative for final version
\newcommand{\sol}[1]{\footnote{#1}\marginpar{\fbox{\thefootnote}}}



\begin{document}
\linenumbers
\begin{frontmatter}

\title{Title: Electricity Price Forecasting using an Efficient Hybridization of Gaussian Processes and Clustering}

\author{A. Yeardley}
\author{D. Roberts}
\author[]{S. Brown \corref{cor_author}}
\ead{s.f.brown@sheffield.ac.uk}

\address{Department of Chemical and Biological Engineering, The University of Sheffield, Sheffield S10 2TN}


\cortext[cor_author]{Corresponding author}

\begin{abstract}

## Intro to try and show the importance of electricity price forecasting
Electricity market price forecasting is becoming increasingly essential for power market players for their decision making. For example, as wholesale electricity prices rise or fall, it gives best times to charge energy storage packs or used energy storage packs. However, electricity demand varies with time naturally, consequentially creating a volatile relationship between electricity price and time. On top of the demand, electricity price can fluctuate due to many other factors such as meteorological conditions making electricity price forecasting increasingly difficult.

Due to the extreme importance of electricity and the risks involved for investment strategies, there are a significant amount of methods that have been used to forecast the price of electricity [one paper gives at least 13 refs]. However, some of the methods have limitations in dealing with sudden changes due to the inflexible nature of statistical models [which ones]. Then machine learning techniques, such as artificial neural networks [refs], do not deal with the uncertainty of the predictions, concentrating instead on point forecasts. Due to the natural volatility of forecasting electricity price, it is important that the users understand the uncertainties involved before making risky decisions.
## show GPs have been successfully used and so we can use it
Whereas, Gaussian processes (GPs) have been used in many fields and have been proven reliably predict the future price of electricity [1], [2] [more refs]. Kou, Liang and Lou [1] used a heteroscedastic Gaussian process with an active learning technique to forecast electricity price for each quarter of 2013. However, the work fails to incorporate the seasonality fluctuation of the electricity price series due to using only subsets of the full data set to train the computationally expensive variational heteroscedastic Gaussian process model. Furthermore, since this research, the smart grid environment has taken a much bigger impact into the electricity pricing structure. “Furthermore, electricity price forecasting in the smart grid environment is also an interesting direction for the future works.”
## mention this paper as it is so much like ours
Combining clustering techniques to Gaussian processes has been shown to improve the accuracy of predictions due to the GPs learning process using similar data points for training. The method was proposed and successfully applied to locational marginal prices in ISO New England, USA [2]. Mori and Nakano compared the uses of Gaussian processes to forecast locational marginal price by using different kernels and various clustering methods. The results found that the use of a deterministic annealing clustering method improves the pre-filtered GP, providing more accurate results. However, this research only uses electricity price data from July 2011 and July 2012 to train the simulation allowing predictions in July 2013. 
## discuss the methodology of our research
This research focuses on the how the clustering method can be used to improve the GP predictive capabilities. Three methods are compared and analysed in order to forecast electricity prices reliably. The first method involves using an ordinary GP as a basis for comparisons while the other two methods involve using clustering methods in two different pre-filtering methods. One way of using clustering is to pre-cluster and condition a single GP so that the cluster number is a new input for the GP to use in its calculations. The second technique is to fit individual GPs for each cluster number. Therefore, GPs are constructed at each cluster number.
This work employs the GPs with the Automatic Relevance Kernel [ref], which conveniently mimics the Gaussian form with a different length-scale for each input dimension. This kernel allows an understanding to the relevance of various subspaces of the input space to the output y. Hence, the relevance of each input is known and so the proportion of output variance ascribable to the input cluster number shows the increased effect having it as an input has on the GP.
Methodology on the clustering technique?
## very fuzzily mention the results
The proposed method is successfully applied to real electricity price data from the United Kingdom [ref]. Comparisons have been made using thorough analysis methods ensuring the predictive capabilities are fully validated. First a cross-validation technique using 5-folds were used on all the training data before the price of electricity data was split at a specific date. Before said date was used by the GPs for learning, then the GPs were used to forecast the price after this date and so the resulting predicted prices and its prediction interval were compared to the true observed prices.
Shall we mention anything about the results? Like which method works best? Would be difficult as we haven’t done it yet.
Shall we discuss how the clustering technique can be used again on the forecasted results to aid in decisions on when to use batteries?


\end{abstract}

\end{frontmatter}

\section*{References}
\bibliographystyle{elsarticle-harv}
\bibliography{MIQP-2019}


\end{document}
